\documentclass[a4paper,11pt]{article}
\usepackage[utf8]{inputenc}
\usepackage[T1]{fontenc}
\usepackage[ngerman]{babel}
\usepackage[margin=2cm]{geometry}
\usepackage{enumitem}
\usepackage{booktabs}
\usepackage{longtable}
\usepackage{tabularx}
\usepackage{xcolor}
\usepackage{fancyhdr}
\usepackage{titlesec}

\titlespacing*{\section}{0pt}{1.2ex}{0.6ex}
\titlespacing*{\subsection}{0pt}{0.8ex}{0.4ex}
\setlength{\parindent}{0pt}
\setlength{\parskip}{0.3em}
\pagestyle{fancy}
\fancyhf{}
\fancyhead[L]{\small BeeZubi Lernwelt GmbH}
\fancyhead[R]{\small Dokumentation -- \texttt{uebersetzer\_programm\_backoff.py}}
\fancyfoot[C]{\thepage}

\begin{document}

\begin{center}
{\Large\bfseries Dokumentation: CSV-"Ubersetzungsprogramm}\\[0.3em]
{\normalsize\texttt{uebersetzer\_programm\_backoff.py}}\\[0.2em]
{\small Stand: Februar 2026}
\end{center}

\section{Zweck}
Das Programm "ubersetzt CSV-Dateien (z.\,B. Pr"ufungsfragen) in mehrere Zielsprachen
mithilfe der OpenAI-API. Es unterst"utzt Fortschritts-Wiederaufnahme, Dedup-Optimierung,
automatische Richtig\_Text-Berechnung und Vollst"andigkeitspr"ufung mit Retry.

\section{Ablauf der "Ubersetzung}

\begin{enumerate}[leftmargin=1.5em, itemsep=1pt]
    \item \textbf{CSV einlesen} -- Die Eingabedatei wird geladen, Spaltennamen normalisiert.
          Optional werden Pr"ufungsspalten umbenannt (\texttt{-{}-pruefung}).
    \item \textbf{Richtig1 korrigieren} -- Zeilen, in denen \texttt{Richtig1} den vollen
          Antworttext statt eines Buchstabens (A--E) enth"alt, werden automatisch korrigiert.
    \item \textbf{Spalten erkennen} -- Textspalten werden erkannt; gesch"utzte Spalten
          (\texttt{-{}-protect-cols}) werden ausgeschlossen. Spalten mit wenigen einzigartigen
          Werten ($\leq 200$) werden als Dedup-Spalten klassifiziert.
    \item \textbf{Pro Zielsprache:}
    \begin{enumerate}[label=\alph*), itemsep=1pt]
        \item Spalte \texttt{Sprache} auf Zielsprache setzen.
        \item \textbf{Dedup-"Ubersetzung} -- Spalten mit wenigen einzigartigen Werten
              (z.\,B. \texttt{LF}, \texttt{Abschnitt}) werden nur 1x pro Wert "ubersetzt.
        \item \textbf{Batch-"Ubersetzung} -- Alle "ubrigen Textspalten werden zeilenweise
              in Batches "ubersetzt. Nach jedem Batch: Zwischenspeicherung + Fortschritt.
        \item \textbf{Richtig\_Text bef"ullen} -- \texttt{Richtig\_Text1}/\texttt{Richtig1\_Text}
              und \texttt{Richtig\_Text2} werden aus der richtigen Antwort (A--E) bef"ullt.
        \item \textbf{Vollst"andigkeitspr"ufung} -- Eingabe und Ausgabe werden zellweise verglichen.
              Falsch-positive Treffer (Zahlen, Codes, Abk"urzungen) werden ignoriert.
        \item \textbf{Retry} -- Verdächtige Zellen werden bis zu 2x neu "ubersetzt.
        \item \textbf{Endg"ultige Speicherung} -- CSV wird gespeichert, Fortschritt auf \texttt{done} gesetzt.
    \end{enumerate}
    \item \textbf{Wiederaufnahme} -- Bei Neustart werden bereits "ubersetzte Dateien nicht
          erneut "ubersetzt. Gesch"utzte Spalten werden aus dem Original wiederhergestellt.
\end{enumerate}

\section{Parameter}

\noindent
\begin{tabularx}{\textwidth}{@{}l l X@{}}
\toprule
\textbf{Parameter} & \textbf{Standard} & \textbf{Beschreibung} \\
\midrule
\texttt{-{}-pdf} & \textit{(Pflicht)} & Eingabe-CSV-Datei (z.\,B. \texttt{EH\_260216.csv}) \\[3pt]
\texttt{-{}-prompt} & \textit{(Pflicht)} & DOCX-Datei mit Megaprompt / "Ubersetzungsregeln \\[3pt]
\texttt{-{}-model} & \texttt{gpt-5-mini} & OpenAI-Modell (z.\,B. \texttt{gpt-4o-mini}) \\[3pt]
\texttt{-{}-temperature} & Modell-Std. & Temperature f"ur die API (z.\,B. 0, 0.3, 1) \\[3pt]
\texttt{-{}-outdir} & \texttt{out\_translated\_backofff} & Ausgabeordner \\[3pt]
\texttt{-{}-langage} & AR TR RU UK PL EN RO & Zielsprachen (Leerzeichen-getrennt) \\[3pt]
\texttt{-{}-batch-size} & 100 & Zeilen pro API-Batch (10--200) \\[3pt]
\texttt{-{}-sep} & \textit{(auto)} & CSV-Trenner (\texttt{;} oder \texttt{,}) \\[3pt]
\texttt{-{}-encoding} & \texttt{utf-8} & Eingabe-Encoding (z.\,B. \texttt{utf-8-sig}) \\[3pt]
\texttt{-{}-protect-cols} & \textit{(siehe unten)} & Gesch"utzte Spalten, die nie "ubersetzt werden \\[3pt]
\texttt{-{}-never-translate} & \textit{(leer)} & Zus"atzliche Spalten, die nie "ubersetzt werden \\[3pt]
\texttt{-{}-dedup-cols} & \texttt{LF,Abschnitt} & Spalten, die immer per Dedup "ubersetzt werden \\[3pt]
\texttt{-{}-column-order} & \textit{(keine)} & Spaltenreihenfolge in der Ausgabe (\texttt{;}-getrennt) \\[3pt]
\texttt{-{}-pruefung} & \textit{(Flag)} & Benennt \texttt{Zwischenpr"ufung} $\to$ \texttt{Abschlusspr"ufung Teil\,1} und \texttt{Abschlusspr"ufung} $\to$ \texttt{Abschlusspr"ufung Teil\,2} um \\
\bottomrule
\end{tabularx}

\textbf{Standard-gesch"utzte Spalten} (\texttt{-{}-protect-cols}):\\
{\small\texttt{lfdNr, FrageNr, BerufNr, Beruf, LFNr, AbschnNr, Nr, Richtig1, Richtig2,
Schwierigkeit, Sprache, Richtig\_Text1, Richtig1\_Text, Richtig\_Text2,
Abschlusspr"ufung\,Teil\,1, Abschlusspr"ufung\,Teil\,2, Lehrjahr, Zwischenpr"ufung, Abschlusspr"ufung}}

\section{Beispielaufruf}

{\small
\begin{verbatim}
python uebersetzer_programm_backoff.py ^
    --pdf EH_260216.csv ^
    --prompt "Megaprompt_26_02_06_MFA_Uebersetzen.docx" ^
    --encoding utf-8-sig ^
    --langage AR TR RU UK PL EN RO ^
    --pruefung
\end{verbatim}
}

\section{Modell und API-Anbindung}

\subsection{Standardmodell: \texttt{gpt-5-mini}}
Das Programm nutzt standardm"a"sig das OpenAI-Modell \texttt{gpt-5-mini} (konfigurierbar "uber
\texttt{-{}-model}). Die Kommunikation erfolgt "uber die OpenAI Chat Completions API.
Pro API-Aufruf wird ein Batch von Zeilen als JSON-Payload gesendet; die Antwort enth"alt
die "ubersetzten Felder im gleichen Schema.

\subsection{Fehlerbehandlung und Backoff}
Bei API-Fehlern (Rate-Limits, Timeouts, Server-Fehler) werden bis zu \textbf{10 Wiederholungsversuche}
mit exponentiellem Backoff durchgef"uhrt ($2^n$ Sekunden, max.\ 60\,s).
Falls das gew"ahlte Modell den \texttt{temperature}-Parameter nicht unterst"utzt
(z.\,B.\ \texttt{gpt-5-mini} akzeptiert nur Standardwert\,1), erfolgt automatisch
ein Fallback ohne \texttt{temperature}.

\subsection{Laufzeit-Kennzahlen pro "Ubersetzungslauf}
Damit die Pipeline sauber skalierbar und reproduzierbar bleibt, werden bei jedem Lauf
die wichtigsten Kennzahlen in der Konsolenausgabe protokolliert:

\noindent
\begin{tabularx}{\textwidth}{@{}l X@{}}
\toprule
\textbf{Kennzahl} & \textbf{Beschreibung} \\
\midrule
Modell & Verwendetes OpenAI-Modell (z.\,B. \texttt{gpt-5-mini}, \texttt{gpt-4o-mini}) \\[3pt]
Batchgr"o"se & Anzahl Zeilen pro API-Aufruf (Standard: 100) \\[3pt]
Temperature & Steuerung der Kreativit"at/Zuf"alligkeit (0\,=\,deterministisch, 1\,=\,kreativ). Ohne Angabe: Modell-Standard \\[3pt]
Anzahl API-Calls & Ergibt sich aus: $\lceil\text{Zeilen} / \text{Batchgr"o"se}\rceil$ pro Sprache\linebreak + Dedup-Aufrufe + ggf.\ Retry-Aufrufe \\[3pt]
Gesamt-Tokens & Abh"angig von Textl"ange und Zielsprache. Wird von der OpenAI-API gemessen und abgerechnet \\[3pt]
Laufzeit & Pro Sprache in der Konsolenausgabe angezeigt (z.\,B. \texttt{FERTIG in 12m34s}).
           ETA wird w"ahrend der Batch-"Ubersetzung laufend berechnet \\
\bottomrule
\end{tabularx}

\medskip
\textbf{Rechenbeispiel:} Eine CSV mit 6\,400 Zeilen, Batchgr"o"se 100, 7~Sprachen ergibt
mindestens $64 \times 7 = 448$ API-Aufrufe (Batch) + Dedup-Aufrufe + ggf.\ Retry.

\section{Ausgabe}
Pro Zielsprache wird eine Datei \texttt{<Eingabename>\_<SPRACHE>.csv} im Ausgabeordner erzeugt
(z.\,B. \texttt{EH\_260216\_AR.csv}). Eine Fortschrittsdatei \texttt{.progress\_<Name>\_<SPRACHE>.json}
erm"oglicht die Wiederaufnahme bei Abbruch.

\end{document}
